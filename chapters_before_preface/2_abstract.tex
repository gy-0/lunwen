\prefacesection{Abstract}

This project presents a holistic system that transforms handwritten text into visually appealing images. Firstly, the image preprocessing stage for the handwritten text is executed to enhance image quality and hence improve downstream text recognition efficiency. A custom-trained Tesseract OCR model is applied to accurately extract the textual content. The extracted text is further processed and transformed into creative prompts through the ChatGPT API, ensuring compliance with the requirements for the generation of images. These prompts are then used by a local image generation model based on Generative Adversarial Networks (GANs) that outputs photorealistic images corresponding to the enriched textual descriptions. The system combines the functionalities of handwriting recognition, language enhancement, and image synthesis, thus effectively bridging textual input and visual output. The project aims to show how visual concepts are transformed in a seamless manner from hand-written characters to pixels, enabling applications in creative content generation, personalized artistic design, and accessible AI-driven design tools. It is also an epitome of what multimodal AI systems can bring, being the combination of OCR, large language models, and GANs, and a practical workflow for the automation of creative processes and enhancement of interaction with technology by users.